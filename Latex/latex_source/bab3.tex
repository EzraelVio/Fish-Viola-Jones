%!TEX root = ./template-skripsi.tex
%-------------------------------------------------------------------------------
%                     BAB III
%               			PEMBAHASAN
%-------------------------------------------------------------------------------

\chapter{METODOLOGI PENELITIAN}

\section{Tahapan Penelitian}

Gambar \textit{flowchart} berikut mengilustrasikan proses pelatihan 
 dari dataset dan juga proses penggunaan yang sesungguhnya.

\begin{figure}[H]
  \centering{}
	\includegraphics[width=0.6\textwidth]{gambar/metode\_pelatihan}
  \caption{Diagram alir untuk algoritma pelatihan klasifikasi objek}
\end{figure}

\begin{figure}[H]
  \centering{}
	\includegraphics[width=0.6\textwidth]{gambar/metode\_use}
  \caption{Diagram alir untuk algoritma pendeteksian objek}
\end{figure}

(Jelasin disini secara ringkas, biar gak terlalu ambigu)

\section{Desain Sistem}

Dalam proses pembuatan \emph{classifier} perlu dilewati tahapan \textit{training}. 
Tujuan \textit{training} adalah untuk menciptakan suatu \emph{strong classifier} 
yang nantinya dapat digunakan untuk melakukan klasifikasi yang sebenarnya.
Pertama, gambar yang akan menjadi \textit{set} latihan dianotasi sesuai kelasnya masing-masing, 
\textit{set} latihan ini juga berisikan gambar-gambar yang tidak memiliki kelas yang benar, atau 
\emph{false example}. Setelah itu gambar melalui proses \emph{pre-processing} dan disesuaikan 
untuk mengoptimalkan proses pelatihan. Setiap gambar latihan nantinya akan 
diubah menjadi \emph{integral image} untuk dapat diproses.

Pertama algoritma akan mengkonstruksi sebuah \emph{decision tree} untuk setiap 
\emph{weaklearn} dimana \emph{branch} untuk setiap kelas akan ditentukan. Lalu 
\emph{Adaboost} akan menentukan bobot dari setiap \emph{weaklearn} dalam menentukan hasil akhir 
klasifikasi gambar. Terakhir, \emph{Adaboost} akan membuat sebuah 
\emph{strong classifier} dengan membuat sebuah \emph{Attentional Cascade}

Dengan \emph{strong classifier}, klasifikasi objek yang sesungguhnya bisa dilakukan. 
Pertama gambar yang akan dideteksi akan melalui \emph{pre-processing}. 
Kemudian \emph{integral image} akan dihitung untuk 
menghasilkan \emph{sub-window} pada gambar. setiap sub-window akan dicek menggunakan 
\emph{strong classifier}. \emph{Sub-window} yang ditemukan memiliki objek yang 
dicari akan dibuatkan \emph{bounding box} pada kelilingnya. Setelah semua 
\emph{sub-window} diperiksa, semua \emph{bounding box} yang bertindihan akan 
digabungkan sebelum gambar hasil dikembalikan ke pengguna.

\section{Training \emph{Strong Classifier}}

Pada \textit{Training} ada tiga tahapan yang perlu dijalankan 
untuk menghasilkan sebuah \emph{Strong Classifier}, 
yaitu penginputan \textit{dataset} pelatihan yang sudah dianotasi, 
pembuatan \emph{decision tree} untuk setiap fitur, 
\emph{Boosting }dan pemilihan fitur oleh Adaboost untuk membuat \emph{attentional cascade}.

\subsection{\textit{Input Dataset} Pelatihan}

\textit{Dataset} yang akan dipakai diambil dari fishR (Dapat dilihat di 
 ini harus diganti) yang 
berisikan gambar ikan genus Abudefduf, Amphiprion dan Chaetodon. 
Untuk contoh pelatihan kelas-kelas ikan adalah gambar berukuran 72x41 piksel, 
dengan warna grayscale. Selain itu juga akan dipilih contoh 
pelatihan negatif atau kumpulan gambar yang tidak terdapat kelas 
ikan dengan jumlah yang sama dari (cari sumber), resolusi sama dan perlakuan 
yang sama. (Jelasin anotasinya kayak gimana disisni) \textit{Dataset} ini lalu 
akan dibagi menjadi tiga yaitu \textit{training dataset}, 
\textit{validation dataset}, dan \emph{test dataset} dengan jumlah 
(belum ditentukan, tergantung dataset. 
Ada banyak metode pembagian yang bisa dipilih).
Nilai bobot pada tiap \textit{dataset} akan di inisialisasi dengan rumus:
\begin{equation}
  w^1_i = D(i) \text{ for } i=1,...,N
\end{equation}
dimana $w$ adalah bobot, $D$ Distribusi dan $N$ jumlah contoh pelatihan.

\subsection{Pembuatan \textit{Decision Tree}}

Setiap \emph{weaklearner} adalah sebuah \emph{decision tree} yang mengambil nilai 
dari \emph{Haar-like features} sebagai variabel klasifikasi.
\emph{Haar-like features} dapat mengembalikan 
sebuah nilai perbandingan intensitas cahaya dari suatu area pada gambar dan 
mendeteksi keberadaan suatu \emph{fitur} seperti 
perbedaan warna, garis, maupun perbedaan kontras pada gambar. 
(Jelasin step by step pembuatan decision tree pake algoritma).

\subsection{\emph{Boosting}}

Algoritma \emph{Adaboost} akan mem-\emph{boosting} \emph{weaklearner} untuk 
menciptakan sebuah \emph{strong classifier}. \emph{Boosting} dilakukan dari 
\emph{weaklearner} yang memiliki tingkat akurasi paling tinggi, dan dengan 
menggunakan \emph{dataset} latihan.

\emph{Adaboost} akan menjalankan \emph{weaklearner} terbaik untuk 
mengklasifikasi seluruh contoh \emph{dataset} latihan dan mencatat contoh yang 
gagal diklasifikasi oleh \emph{weaklearner} tersebut. Nilai dari \emph{weaklearner} 
tersebut dihitung dengan (Masukin rumus disini, yg gini purity itu), yang nantinya 
akan dicatat untuk urutan \emph{Boosting} ronde berikutnya dan juga sebagai nilai 
voting pada klasifikasi yang sebenarnya. 
Nantinya bobot dari seluruh \emph{dataset} latihan akan dihitung ulang dengan menaikan bobot dari contoh 
yang gagal diklasifikasi oleh \emph{weaklearner} sebelumnya. Proses ini dilakukan 
untuk setiap \emph{weaklearner}. (Masukin rumus re-distribusi bobot contoh disini)

Ketika sebuah ronde \emph{Boosting} sudah selesai dan seluruh \emph{weaklearner} sudah  diurutkan sesuai dengan 
nilai yang didapat dari ronde tersebut. \emph{Adaboost} akan melakukan validasi 
tingkat akurasi \emph{strong classifier} yang sudah dibuat menggunakan \emph{dataset} validasi. 
hal ini dilakukan dengan membandingkan tingkat presisi \emph{strong classifer} 
yang dihasilkan ronde ini dengan \emph{strong classifer} yang dihasilkan ronde 
sebelumnya. Bisa tingkat presisi ternyata semakin berkurang, maka \emph{Boosting} 
akan dihentikan dan \emph{strong classifier} (ronde ini atau sebelumnya, cek lagi) 
menjadi \emph{final strong classifier}.

\subsection{Pembuatan \emph{Attentional Cascade}}

Sebelum pembuatan \emph{attentional cascade}, \emph{weaklearner} yang kurang 
diskriminatif akan dibuang. Hal ini dilakukan dengan (cari, karena bagian ini 
belum jelas dari kemarin). \emph{Attentional cascade} akan dibuat dari \emph{weaklearner} 
yang tersisa secara bertahap. Pada setiap fase \emph{cascade}, \emph{weaklearner} akan 
ditambahkan satu-persatu. Setiap sebuah \emph{weaklearner} ditambahkan, tes akan dilakukan 
dengan \emph{dataset} tes. \emph{Weaklearner} akan terus ditambahkan hingga \emph{false positive rate} 
yang ditentukan untuk fase itu dicapai. Fase akan terus bertambah hingga 
akurasi sempurna dicapai, atau hingga \emph{weaklearner} sudah habis. (Ini periu 
dijabarin lagi dengan rumus per-step).

\section{Klasifikasi Objek Sebenarnya}

Tahapan ini adalah penggunakaan \emph{classifier} yang sebenarnya. 
Gambar yang akan dipakai akan melalui beberapa langkah dalam tahap ini, 
yaitu: \textit{Pre-processing} dalam bentuk \emph{grayscaling}, Penghitungan 
\emph{Integral Image}, dan deteksi yang sesungguhnya menggunakan \emph{strong classsifier} 
yang telah dibuat dengan metode Sliding Window.

\subsection{\textit{Pre-processing} dan Penghitungan \emph{Integral Image}}

Untuk klasifikasi sebenarnya, gambar \emph{input} akan diproses terlebih dahulu. 
Gambar awalnya akan melakui proses \textit{pre-processing} dan dirubah kedalam warna 
\emph{grayscale} (Jelasin perubahan greyscale ini kayak gimana) 
untuk mempermudah penghitungan dengan bantuan \emph{library OpenCV}. 
Setelah tu sebuah matriks sebesar resolusi gambar akan dibuat untuk 
penghitungan \emph{Integral Image} yang nantinya akan mempercepat proses 
penghitungan fitur.

Untuk pembuatan \emph{Integral Image}, pertama perlu dicari nilai fitur 
baris pertama dan kolom pertama dari gambar. Nilai pada matriks \emph{Integral Image} 
adalah median dari nilai RGB pada piksel tersebut, namun dikarenakan warna 
gambar sudah dirubah menjadi \emph{greyscale} maka nilai pada setiap piksel 
hanya akan berupa bilangan bulat dikisaran 0 sampai 255. Nilai pada kolom 
pertama hanyalah penjumlahan nilai piksel dari piksel $(0, 0)$ sampai ke piksel 
$(0, j)$, sementara nilai pada baris pertama juga hanya penjumlahan nilai 
piksel dari $(0, 0)$ ke  $(i, 0)$. Nilai-nilai piksel lainnya lalu dapat 
dihitung menggunakan rumus: 
\begin{equation}
  \begin{split}
    \text{integral image } (i,j) = {} & \text{integral image } (i-1,j) + \text{integral image } (i,j-1) - \\
    & \text{integral image } (i-1,j-1) + \text{nilai dari piksel } (i,j)
  \end{split}
\end{equation}
Atau jumlah semua nilai piksel dari $(0, 0)$ sampai ke $(i, j)$.
Pembuatan \emph{integral image} nantinya akan mempercepat proses penghitungan 
fitur pada setiap \emph{sub-window}.


\subsection{Klasifikasi}


