%!TEX root = ./template-skripsi.tex
%-------------------------------------------------------------------------------
%                            	BAB IV
%               		KESIMPULAN DAN SARAN
%-------------------------------------------------------------------------------

\chapter{KESIMPULAN DAN SARAN}

\section{Kesimpulan}
Berdasarkan hasil implementasi yang telah dilakukan penulis serta pengujian fitur klasifikasi yang 
telah penulis rancang, maka didapatkan kesimpulan sebagai berikut:

\begin{enumerate}
	\item Dibuatnya \textit{classifier} spesies ikan berbasis Viola-Jones Feature Extraction dan Boosting
	Berbasis Decision Tree. Adapun perancangan \textit{classifier} ini dikerjakan dalam 
	waktu kurang lebih satu tahun.

	\item Berdasarkan hasil pengetesan akurasi yang dilakukan pada saat validasi, metode ini 
	berhasil mengklasifikasi 12 dari 75 gambar, atau memiliki akurasi 16\%.

	\item Klasifikasi memiliki akurasi rendah dikarenakan dua faktor, yaitu tidak 
	sesuainya lokasi \emph{sub-window} yang diklasifikasi dan juga bias bobot \textit{voting} 
	\emph{weak classifier} awal pada \emph{cascade}. Bias nampaknya ditunjukan kepada kelas 
	ikan Abudefduf.

\end{enumerate}

\section{Saran}
\begin{enumerate}
	\item Penelitian ini masih belum bisa secara akurat mengklasifikasi jenis ikan. Oleh 
	karena itu perlu ada penelitian lanjutan yang menyempurnakan tingkat akurasi klasifikasi 
	dengan mengganti atau merombak metode yang ada, terutama metode \emph{sliding window} 
	dan bias klasifikasi ke kelas tertentu.
	\item Pemutakhiran dapat dilakukan agar dapat menerima gambar dengan ukuran yang 
	dinamis 
	\item Pengimplementasian \textit{multiprocessing} untuk mempercepat proses 
	\textit{training} agar memungkinkan training dengan data yang lebih besar.
\end{enumerate}

% Baris ini digunakan untuk membantu dalam melakukan sitasi
% Karena diapit dengan comment, maka baris ini akan diabaikan
% oleh compiler LaTeX.
\begin{comment}
\bibliography{daftar-pustaka}
\end{comment}
