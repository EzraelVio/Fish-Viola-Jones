%!TEX root = ./template-skripsi.tex
%-------------------------------------------------------------------------------
%                            BAB II
%               KAJIAN TEORI
%-------------------------------------------------------------------------------

\chapter{KAJIAN PUSTAKA}

\section{Pengertian Klasifikasi Objek}

Fungsi dari klasifikasi objek adalah untuk memberikan deskripsi label ke 
sebuah segmentasi objek. Bila dilihat dari representasi fitur objek, ini 
dapat dicapai dengan melihat tanda-tanda keberadaan fitur yang mengindikasikan 
kelas dari objek. Hal ini umumnya dicapai dengan menentukan batasan diantara 
kelas-kelas yang direpresentasikan didalam \textit{training set} yang sudah diberi label. 
Pelatihan dilakukan dengan secara data yang sudah dilabeli secara manual atau 
data yang belum dilabeli. Pelatihan otomatis umumnya menekankan distribusi dari 
setiap kelas, dan melabeli setiap contoh dengan sesuai, Namun, isu yang umum 
didalam kedua kasus adalah pilihan dari fitur-fitur yang akan digunakan untuk 
klasifikasi objek (Renno et al, 2007).

\section{\emph{Viola Jones Object Detection Framework}}

Paul Viola dan Michael, J, Jones mempublikasikan sebuah makalah ilmiah dengan 
judul “\emph{Robust Real-Time Face Detection}”. Makalah tersebut mendeskripsikan sebuah 
framework pendeteksian wajah yang dapat memproses gambar secara cepat dengan 
tingkat akurasi yang tinggi. Ada tiga kontribusi penting dari makalah tersebut: 
Pertama adalah pengenalan sebuah representasi gambar baru yang dipanggil 
\emph{Integral Image} yang memungkinkan penghitungan fitur yang dipakai oleh detektor 
dilakukan dengan cepat. Yang kedua adalah sebuah classifier yang efisien dan 
sederhana, yang dibuat menggunakan algoritma pembelajaran \emph{Adaboost} 
(Freund et al, 1995) untuk memilih sejumlah fitur-fitur kritis dari 
fitur-fitur potensial. Yang ketiga adalah sebuah metode untuk menggabungkan 
fitur-fitur tersebut dalam sebuah bentuk \emph{cascade} yang memungkinakan algoritma 
untuk memfokuskan deteksi di area-area yang memiliki potensial saja.

\subsection{\emph{Features}}

\begin{figure}[H]
  \centering{}
	\includegraphics[width=0.6\textwidth]{gambar/haar\_features}
  \caption{Beberapa \emph{Haar-like features} yang digunakan framework Viola-Jones.}
\end{figure}

Ada banyak alasan dimana penggunaan fitur lebih baik daripada 
nilai piksel secara langsung. Alasan paling umum adalah fitur dapat berfungsi
untuk meng-\emph{encode ad-hoc domain knowledge} yang sulit dipelajari 
dengan jumlah data pelatihan yang terbatas. Untuk sistem \emph{framework Viola-Jones} 
ada alasan besar lainnya untuk penggunaan fitur yaitu sistem berbasis fitur beroperasi lebih cepat
daripada sistem yang berbasis nilai piksel.

Fitur sederhana yang digunakan mirip dengan \emph{Haar Basis Function} yang 
digunakan Papageorgiou et al. (1998). Lebih tepatnya, tiga jenis fitur. 
Nilai dari sebuah \emph{fitur dua persegi} adalah perbedaaan diantara 
jumlah nilai piksel didalam dua area persegi. Area-area tersebut memiliki 
ukuran dan bentuk yang sama, dan juga bersebelahan secara horizontal dan vertikal.
Subuah \emph{fitur tiga persegi} menghitung jumlah piksel dua area persegi 
di bagian luar dikurangi dengan jumlah nilai piksel persegi yang ada ditengah keduanya. 
Yang terakhir \emph{fitur empat persegi} menghitung perbedaan nilai dari dua pasang 
persegi diagonal.

\subsection{\emph{Integral Image}}

\begin{figure}[H]
  \centering{}
	\includegraphics[width=0.6\textwidth]{gambar/integral\_image\_1}
  \caption{Sebuah nilai \emph{Integral Image} $(x,y)$ dan area yang diwakilinyak}
\end{figure}

Fitur-fitur persegi dapat dihitung secara cepat menggunakan 
representasi tidak langsung dari gambar, hal ini diberi nama 
\emph{integral image}. \emph{integral image} pada lokasi $x, y$ berisikan 
penjumlahan di atas dan di kiri dari $x, y$ dan nilai $x, y$ itu sendiri:
\begin{equation}
  i i(x, y)=\sum_{x^{\prime} \leq x, y^{\prime} \leq y} i\left(x^{\prime}, y^{\prime}\right),
\end{equation}
Dimana $ii(x,y)$ adalah \emph{integral image} dan $i(x,y)$ 
adalah nilai piksel dari gambar aslinya. Menggunakan kedua 
pengulangan berikut ini:
\begin{equation}
  s(x, y)=s(x, y-1) + i(x, y)
\end{equation}
\begin{equation}
  ii(x, y)=ii(x-1, y) + s(x, y)
\end{equation}
(Dimana \emph{s(x, y)} adalah nilai kumulatif dari baris, \emph{s(x, -1) = 0}, 
dan \emph{ii(-1, y) = 0}) \emph{Integral Image} dari sebuah gambar dapat dihitung dalam sekali jalan.

\begin{figure}[H]
  \centering{}
	\includegraphics[width=0.6\textwidth]{gambar/integral\_image\_2}
  \caption{Jumlah dari intensitas cahaya pada persegi D dapat dihitung dengan 4 referensi \textit{array}. Nilai dari 1 adalah jumlah intensitas cahaya pada persegi A, Nilai dari 2 adalah persegi A + B, nilai dari 3 adalah A + C dan nilai 4 adalah A + B + C + D.}
\end{figure}

Menggunakan \emph{integral image}, semua jumlah nilai pada persegi dapat 
dihitung didalam 4 referensi \emph{array}. Jelas perbedaan 
diantara kedua jumlah nilai-nilai persegi dapat dihitung 
dengan delapan referensi. Karena peresegi dua fitur yang 
didefinisikan diatas melibatkan juga nilai persegi disebelahnya, 
mereka dapat dikomputasi dengan enam referensi \emph{array}, delapan referensi
bilamana ia adalah persegi tiga fitur, dan sembilan referensi untuk 
persegi empat fitur.

Kecepatan yang didapat dari penghitungan \emph{Integral Image} ini dapat dijustifikasi 
bila kita membandingkannya dengan perhitungan manual. Sebagai contoh, misalkan 
kita sedang mencari jumlah total intensitas cahaya pada ukuran area 10x10 
piksel. Cara manual mengharuskan kita menghitung sampai 100 kali untuk mendapat 
jumlah intensitas cahaya pada area tersebut, belum lagi proses ini harus 
diulang terus-menerus untuk ukuran dan lokasi yang berbeda. Dilain sisi, 
perhitungan menggunakan \emph{Integral Image} hanya perlu mereferensi tabel yang 
sudah dibuat sebelum semua usaha klasifikasi, dalam hal ini kita hanya perlu 
melakukan perhitungan empat kali untuk menghitung intensitas cahaya dalam area 
tersebut.

\subsection{\emph{Adaboost}}

\begin{algorithm}
  \caption{Algoritma Adaboost}
  \begin{algorithmic} [1]
    \State Diberikan contoh gambar $(x_1, y_1),...,(x_n, y_n)$ dimana 
    $y_i = 0$ untuk contoh negatif dan $y_i = 1$ untuk contoh positif
    \State Inisialisasi bobot $w_1,_i = \frac{1}{2_m}, \frac{1}{2l}$ 
    untuk $y_i = 0$ dan $y_i = 1$, dimana $m$ adalah angka negatif dan 
    $l$ adalah angka positif
    \State Untuk $t = 1,...,T$:
      \begin{enumerate}
        \item Normalisasi bobot, $w_t,_1 \gets \frac{w_t,_1}{\sum_{j=1}^{n} w_t,_j}$
        \item Pilih $weak classifier$ terbaik dengan melihat error yang telah diberi bobot
          \begin{equation}
            \in_t = min_f,_p,_\theta \sum_i w_i|h(x_i,f,p,\theta) - y_i|.
          \end{equation}
        \item Definisikan $h_t(x) = h(x,f_t,p_t,\theta_t)$ dimana $f_t,p_t,$ dan $\theta_t$ adalah \textit{minimizer} dari $\in_t$.
        \item Perbarui bobot:
          \begin{equation}
            w_t+1,i = w_t,i \beta{t}{1-e_i}
          \end{equation}
        dimana $e_i = 0$ bila contoh $x_i$ diklasifikasi secara benar, selainnya $e_i = 1$, dan $\beta_t = \frac{\in_t}{1-\in_t}$
      \end{enumerate}
      \State \textit{strong classifier} akhirnya adalah:
        \begin{equation}
          C(x) = \begin{cases}
            1 & \sum\limits_{t=1}^T \alpha_t h_t (x) \geq \frac{1}{2} \sum\limits_{t=1}^T \alpha_t \\
            0 & \text{selain itu}
          \end{cases}
        \end{equation}
      dimana $\alpha_t$ = log $\frac{1}{\beta_t}$ 
  \end{algorithmic}
\end{algorithm}

Dalam \emph{framework Viola-Jones} sebuah varian dari \emph{Adaboost} 
digunakan untuk memilih fitur dan juga untuk melatih \textit{classifier}. 
Didalam bentuk aslinya, algoritma pembelajaran \emph{Adaboost} digunakan 
untuk mem-\textit{boost} performa klasifikasi dari algoritma pembelajaran 
sederhana. \emph{Adaboost} melakukan ini dengan menggabungkan sekumpulan 
\emph{classifier} lemah untuk membuat sebuah \emph{classifier} kuat. 
Didalam istilah \emph{Boosting}, \emph{classifier} lemah disebut juga 
dengan \emph{weak learner}. Sebagai contoh, algoritma pembelajaran \emph{perceptron} 
mencari dari sekelompok \emph{perceptron} dan mengambil \emph{perceptron} 
dengan tingkat kesalahan klasifikasi terendah. Algoritma pembelajaran disebut 
lemah karena kita tidak berharap \emph{classifier} terbaik untuk mengklasifikasi 
data dengan benar. Nyatanya, \emph{perceptron} terbaik mungkin 
hanya memiliki tingkat akurasi 51\%. Agar \emph{weak learner} dapat di-\textit{boost}, 
ia akan dipanggil untuk menyelesaikan sederet problem pembelajaran. Setelah 
satu ronde pembelajaran selesai, contoh pembelajarannya akan dibobot ulang 
untuk menekankan problem yang salah diklasifikasi oleh \emph{weak learner} 
sebelumnya. Bentuk \emph{final strong classifier} adalah \emph{perceptron}, 
sebuah kombinasi \emph{weak learner} berbobot yang diikuti oleh \emph{threshold}.

Dalam \emph{Framework Viola Jones}, \emph{Adaboost} yang digunakan hanya bekerja untuk 
klasifikasi biner saja. Oleh karena itu perlu adanya modifikasi \emph{Adaboost} yang 
dapat melakukan klasifikasi \emph{Multi-class}. Yang pertama adalah anotasi 
contoh gambar $(x_1, y_1),...,(x_n, y_n)$ dimana $y_i \in Y = {1,...,k}$. 
Maka dari itu distribusi $(D)$ nilai bobot $(w)$ akan berubah sesuai dengan 
jumlah contoh $(N)$: $w^1_i = D(i) \text{ for } i=1,...,N$. %(harus dilanjutin)

\subsection{\emph{Weaklearn}}

\begin{algorithm}
  \caption{Metode Pembuatan \textit{Decision Tree}}
  \begin{algorithmic} [1]
    \State Anotasi semua dataset sesuai kelasnya. \textit{Decision Tree} lalu 
    akan dimulai dari akar
    \State Pilih atribut terbaik untuk melakukan \textit{split} dengan melakukan 
    \emph{Information Gain} (IG):
    \begin{equation}
      IG(S, A) = Entropi(S) - sum_v \frac{S_v}{S} x Entropi(S_v)
    \end{equation} 
    dimana:
    \begin{itemize}
      \item $S$: kumpulan sampel pada \emph{node} sekarang
      \item $A$: atribut yang digunakan untuk \textit{split}
      \item $v$: Sebuah nilai atribut $A$ yang membagi $S$ menjadi turunan $S_v$
      \item Entropi($S$) = $-sum_c p_c log_2(p_c)$: Ukuran kemurnian pada kumpulan 
      contoh dengan target label. Dimana $p_c$ adalah proporsi contoh $S$ yang ada 
      pada kelas $c$. 
    \end{itemize}
    %Masukin IGR kalo perlu (Information Gain Ratio)
    \State \textit{split} pohon ke \textit{node} turunan sesuai atribut yang dipilih 
    \State tentukan apabila seluruh contoh sudah jatuh ke kelas yang benar, bila tidak 
    maka ulangi langkah 2 dan 3. Hal ini dilakukan dengan memvalidasi pohon  yang 
    sudah dibuat
  \end{algorithmic}
\end{algorithm}

\emph{Framework Viola Jones} menggunakan sebuah \emph{weaklearn} yang bernama 
\emph{Decision Stump}, atau sebuah \emph{Decision Tree} yang hanya memiliki 
dua daun kelas saja. \emph{Decision Tree} sendiri mampu digunakan untuk 
permasalahan \emph{multi-class}.

\emph{Decision Tree} menghasilkan sebuah classifier didalam bentuk sebuah pohon 
pilihan, sebuah struktur yang berbentuk:
\begin{itemize}
  \item Sebuah daun, mengindikasi sebuah kelas, atau
  \item Sebuah \emph{decision node} yang menspesifikasi 
  sebagian tes untuk dikerjakan atas sebuah nilai 
  atribut, dengan satu \emph{branch} dan \emph{subtree} untuk 
  setiap hasil dari tes.
\end{itemize}

Sebuah \emph{Decision Tree} dapat digunakan untuk mengkasifikasi sebuah kasus 
dengan memulai dari akar pohon dan bergerak sampai sebuah daun ditemukan. 
Pada setiap \emph{node} yang bukan merupakan daun, hasil dari tes kasus 
dideterminasi dan perhatian berubah ke akar dari \emph{subtree} sesuai 
dengan hasil tersebut. Ketika proses pada akhirnya (dan dengan pasti) 
menuju ke sebuah daun, kelas dari kasus diprediksi sesuai yang ada di daun.

\subsection{\emph{Attentional Cascade}}

\emph{Attentional Cascade} adalah sebuah \emph{cascade} dari banyak 
\emph{classifier} yang dibuat untuk meningkatkan performa deteksi dengan secara radikal mengurangi waktu 
komputasi. Intinya \emph{classifier} kecil yang telah di-\emph{boost} dapat dibuat lebih 
kecil dan efisien, yang dapat menolak mayoritas \emph{sub-window} negatif dan 
mendeteksi sebagian besar dari \emph{sub-window} positif. \emph{Classifier} yang lebih 
sederhana digunakan untuk menolak mayoritas \emph{sub-window} sebelum \emph{classifier} 
yang lebih kompleks dipanggil untuk menurunkan tingkat \emph{false positives}.

Struktur dari \emph{cascade} merefleksikan 
fakta bahwa pada gambar apapun mayoritas \emph{sub-window} pasti negatif. 
Oleh karena itu, \emph{cascade} berusaha untuk sebanyaknya menolak sub-window 
negatif pada tahapan seawal mungkin. Sementara hasil positif 
akan memicu evaluasi dari setiap \emph{classifier} dalam \emph{cascade}, 
hal ini sangatlah langka.

Layaknya sebuah \emph{Decision Tree}, sebuah \emph{classifier} dilatih menggunakan 
contoh-contoh yang telah berhasil melewati tahap sebelumnya. Oleh karenanya, 
\emph{classifier} pada tahap kedua menghadapi tantangan yang jauh lebih sulit 
daripada yang pertama. Contoh yang semakin sulit yang dihadapi \emph{classifier} 
di tahap-tahap yang semakin jauh menekan seluruh \emph{receiver operating characteristic} (ROC) 
kebawah.

\begin{algorithm}
  \caption{Algoritma Pelatihan Untuk Pembuatan \textit{Cascaded Detector}}
  \begin{algorithmic} [1]
    \State Pengguna memilih nilai dari \textit{f}, nilai maksimum \emph{false positive} 
    pada setiap tahap yang dapat diterima, dan \textit{d}, nilai minimum \emph{detection rate} 
    per tahap yang dapat diterima
    \State Pengguna memilih target keseluruhan \emph{false positive rate}, $F_{target}$
    \State \textit{P} = kumpulan contoh positif
    \State \textit{N} = kumpulan contoh negatif
    \State $F_0 = 1.0; D_0 - 1.0$
    \State $i = 0$
    \State $while F_i > F_{target}$
    \Statex $-i \leftarrow i + 1$
    \Statex $-n_i = 0; F_i = F_i-_1$
    \Statex -while $F_i > f x F_i-_1$
    \begin{itemize}
      \item $n_i \leftarrow n_i + 1$
      \item Gunakan \textit{P} dan \textit{N} untuk melatih sebuah \emph{classifier}
      dengan fitur $n_i$ menggunakan \emph{Adaboost} 
      \item Evaluasi \emph{cascade classifier} dengan \textit{set} validasi
      untuk menentukan $F_i$ dan $D_i$. 
      \item kurangi \textit{threshold} untuk \emph{classifier} ke-$i$ sampai 
      \emph{cascade classifier} memiliki tingkat deteksi paling tidak $d x D_i-_1$ 
      (hal ini juga akan mempengaruhi $F_i$)
    \end{itemize}
    \Statex $-N \leftarrow$ {\O}
    \Statex -if $F_i > F_target$ evaluasi \emph{cascade detector} menggunakan 
    \textit{set} negatif dan masukan semua deteksi gagal ke \textit{set} N
  \end{algorithmic}
\end{algorithm}

