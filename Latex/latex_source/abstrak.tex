\chapter*{\centering{\large{ABSTRAK}}}
\singlespacing{}

\textbf{Nehemiah Austen Pison}, \textit{Protptype System} Pendeteksi Spesies Ikan 
Menggunakan \textit{Viola-Jones Feature Extraction} dan \textit{Boosting} Berbasis 
\textit{Decision Tree}. Skripsi, Program Studi Ilmu Komputer, Fakultas Matematika dan Ilmu Pengetahuan Alam, Universitas Negeri Jakarta. Januari 2024.
\\
\\
Sektor perikanan di Indonesia merupakan pasar yang sangat besar dengan nilai 
mencapai Rp 2.400 triliun pada tahun 2022. Meskipun begitu, potensi perikanan 
di Indonesia belum sepenuhnya terealisasi. Dalam industri perikanan ada problematika 
dalam penghitungan dan klasifikasi ikan, ikan masih dihitung secara manual satu-persatu 
atau dengan menghitungnya dengan wadah. Metode-metode lain yang menggunakan mesin 
juga punya hambatannya, seperti spesifiknya ukuran ikan yang bisa dideteksi, Dalam 
penelitian ini kami mengusulkan metode klasifikasi ikan menggunakan metode \emph{Viola-Jones Feature Extraction} dan \textit{Boosting} Berbasis 
\emph{Decision Tree} untuk menyelesaikan masalah perhitungan maupun klasifikasi ikan 
untuk semua ukuran. Dengan menggunakan klasifikasi komputer, pendeteksian tidak terhambat 
oleh ukuran, maupun bentuk ikan yang berbeda-beda. Metode ini juga memiliki potensi 
klasifikasi jenis ikan. Langkah pertama adalah dengan melatih \emph{weak classifier} 
untuk mengklasifikasi beberapa fitur beberapa jenis ikan, diikuti dengan \emph{Boosting}, 
dan pembuatan sebuah \emph{cascade} untuk mempercepat proses klasifikasi. Klasifikasi 
ini diharapkan dapat membantu dalam proses penghitungan ikan, serta klasifikasi ikan 
bagi para petani ikan maupun umum.
\\
\\
\textbf{Kata kunci}: klasifikasi, Viola-Jones, Boosting, Decision Tree, Ikan.