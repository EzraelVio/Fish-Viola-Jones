\chapter*{\centering{\large{ABSTRAK}}}
\singlespacing{}

\textbf{Nehemiah Austen Pison}, \textit{Protptype System} Pendeteksi Spesies Ikan 
Menggunakan \textit{Viola-Jones Feature Extraction} dan \textit{Boosting} Berbasis 
\textit{Decision Tree}. Skripsi, Program Studi Ilmu Komputer, Fakultas Matematika dan Ilmu Pengetahuan Alam, Universitas Negeri Jakarta. Januari 2024.
\\
\\
Sektor peternakan perikanan di Indonesia memiliki problem yang lumayan besar dalam penghitungan. 
Yaitu ikan masih dihitung satu per satu atau dengan metode-metode lainnya yang kurang akurat. 
Metode lainnya yang menggunakan mesin juga punya problemnya tersendiri, yaitu limitasi 
spesifikasi ukuran ikan yang bisa dihitung. Problem penghitungan ini sangat besar dalam 
industri peternakan ikan di Indonesia yang sangat mementingkan jumlah ikan dalam budidaya. 
Dalam penelitian ini, penulis menggunakan 
metode \emph{Viola-Jones Feature Extraction} dan \textit{Boosting} Berbasis 
\emph{Decision Tree} untuk melakukan klasifikasi ikan, yang kedepannya dapat digunakan untuk 
melakukan penghitungan ikan. Menggunakan klasifikasi komputer sebagai basis penghitungan 
memungkinkan ikan dengan ukuran beda, maupun spesies berbeda untuk dihitung. 
Langkah pertama dalam metode ini adalah dengan melatih \emph{weak classifier} 
untuk mengklasifikasi beberapa fitur beberapa kelas ikan, diikuti dengan \emph{Boosting}, 
dan pembuatan sebuah \emph{cascade} untuk mempercepat proses klasifikasi. Klasifikasi 
ini diharapkan dapat membantu dalam proses penghitungan ikan, serta klasifikasi ikan 
bagi para petani ikan maupun umum. 
\\
\\
\textbf{Kata kunci}: klasifikasi, Viola-Jones, Boosting, Decision Tree, Ikan.