\chapter*{\centering{\large{ABSTRAK}}}
\singlespacing{}

\textbf{Nehemiah Austen Pison}, \textit{Protptype System} Pendeteksi Spesies Ikan 
Menggunakan \textit{Viola-Jones Feature Extraction} dan \textit{Boosting} Berbasis 
\textit{Decision Tree}. Skripsi, Program Studi Ilmu Komputer, Fakultas Matematika dan Ilmu Pengetahuan Alam, Universitas Negeri Jakarta. Januari 2024.
\\
\\


Luka kronis merupakan masalah kesehatan yang kompleks, khususnya bagi pasien dengan 
penyakit seperti Diabetes Melitus (DM). Proses penyembuhan luka melibatkan asesmen 
yang cermat dan pengelolaan yang efektif, namun metode manual dalam pengukuran 
luka seringkali tidak akurat dan memakan waktu. Dalam penelitian ini, kami mengusulkan 
penggunaan metode \emph{GrabCut} dalam pemrosesan citra untuk mendeteksi area keliling 
luka kronis. Metode ini memiliki potensi untuk memberikan analisis yang objektif 
dan reliabel dalam asesmen luka. Dengan memanfaatkan teknologi pemrosesan gambar dan pembelajaran mesin, kami mencoba 
mengatasi keterbatasan metode manual dengan menggunakan algoritma \emph{GrabCut}. Langkah 
pertama melibatkan pengambilan foto luka menggunakan perangkat seluler, diikuti 
oleh kalibrasi warna untuk memastikan konsistensi. Metode \emph{GrabCut} kemudian diterapkan 
untuk mengklasifikasikan area luka, dengan fokus pada pemisahan antara objek utama 
(luka) dan latar belakang. Penelitian ini menggunakan dataset citra luka kronis dari penelitian sebelumnya 
dan berupaya membandingkan hasil metode \emph{GrabCut} dengan citra referensi sebagai 
\emph{ground truth}. Diharapkan bahwa penelitian ini akan memberikan kontribusi 
terhadap efektivitas asesmen luka kronis dan mempercepat proses pengelolaan, serta 
membantu mengurangi beban biaya bagi pasien.
\\
\\
\textbf{Kata kunci:} luka kronis, \emph{GrabCut}, pemrosesan citra, asesmen luka, pengelolaan luka.