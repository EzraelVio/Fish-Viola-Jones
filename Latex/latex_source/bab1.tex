%!TEX root = ./template-skripsi.tex
%-------------------------------------------------------------------------------
% 								BAB I
% 							LATAR BELAKANG
%-------------------------------------------------------------------------------

\chapter{PENDAHULUAN}

\section{Latar Belakang Masalah}

% Intro
Sektor ikan di Indonesia adalah sebuah pasar yang sangat besar, nilainya mencapai 
Rp 2.400.000.000.000.000 rupiah pada tahun 2022. Meskipun pasar yang besar, potensi 
perikanan di Indonesia belum maksmimal, tutur Menteri Kelautan dan Perikanan RI Wahyu Sakti Trenggono. 
Namun produktivitas perikanan di Indonesia masih kalah dari Vietnam. 
Ia menegaskan bahwa potensial perikanan tersebut harus ditangkap dengan baik terutama 
di sektor budidaya yang saat ini berjalan dalam sekala kecil, berbeda dengan negara 
tetangga seperti Vietnam yang mengusai eksport ke berbagai negara (\cite{harianjogja}).

% Dasar problem permasalahan lapangan
Budidaya ikan di Indonesia memiliki problem yang lumayan besar yaitu 
diperlukannya usaha yang besar untuk menghitung dan mengawasi jumlah ikan 
yang dibudidayakan. Dalam penghitungan bibit pada proses jual beli contohnya, 
dalam menghitungan bibit lele para pedagang masih menghitung ikan dengan 
cara manual (\cite{alamri}). Bibit ikan dipindahkan satu persatu atau 
ditimbang sesuai berat untuk mendapatkan jumlah ikan. kedua cara tersebut antara 
sangat tidak efisien atau kurang akurat. Dalam metode penghitungan, ikan 
dihitung satu per satu dengan tangan atau dengan bantuan sendok atau centong, 
yang memungkinkan penghitung untuk mengambil ikan dengan jumlah tertentu. 
Metode ini bisa memakan waktu yang cukup lama bila ikan yang dihitung 
berjumlah besar, karena itu metode ini biasanya hanya digunakan 
dalam menghitung ikan dalam jumlah yang sedikit. Sementara itu metode 
penimbangan hanya menghasilkan jumlah perkiraan yang tidak 
selalu akurat, namun cepat. Dalam metode ini bibit ikan dimasukan ke dalam suatu wadah 
yang lalu ditimbang. Berat hasil penimbangan lalu bisa 
dijadikan acuan kira-kira jumlah ikan yang terdapat di dalam wadah. 
Metode ini cepat dan cukup efisien dalam menghitung ikan dalam jumlah yang 
sangat besar. Namun jumlah bibit ikan tidaklah akurat dan hanya berupa perkiraan belaka.

% Elaborasi lebih lanjut problem lapangan
Problem perhitungan ikan ini akan sangat terasa pada industri budidaya ikan yang 
sangat mementingkan kepadatan populasi dalam tempat budidaya ikan. Populasi ikan 
yang berlebihan dapat memperlambat pertumbuhan ikan (\cite{diansarietal}), tetapi 
di sisi lain populasi ikan yang terlalu kecil akan mengurangi efisiensi lahan 
yang dimiliki peternak ikan. Dalam mengatasi problem \cite{alamri} menciptakan 
sebuah sistem penghitungan menggunakan sensor \emph{proximity}. Hasil uji coba mendapat 
hasil yang baik dengan persentase error sebesar 4,07\% dengan penghitungan 
memakan waktu 228 detik per 1000 ikan. Jauh lebih cepat dibanding kecepatan 
hitung manual yang memakan waktu 20 menit per 1000 ikan. Cara lain dipakai 
oleh \cite{rusydi} untuk mendeteksi ikan. Rusydi menciptakan sebuah alat 
penghitungan dengan katup otomatis yang akan terbuka bilamana jumlah ikan 
yang diinginkan telah tercapai. Alat tersebut mendeteksi ikan menggunakan 
konsep \emph{through beam} di mana ikan akan terdeteksi ketika melewati pipa oleh 
inframerah dan photodioda. Alat tersebut dapat mendeteksi ikan dengan 
kecepatan 58 ms per ikan dengan tingkat akurasi 100\%.

% Kelemahan metode penghitungan manual Al-Amri dan Rusydi
Penggunaan alat deteksi fisik seperti yang digunakan \cite{alamri} maupun 
\cite{rusydi} memiliki beberapa kekurangan, seperti ukuran ikan bergantung 
kepada ukuran alat yang dipergunakan. Alat Rusydi sangat bergantung 
dengan kelandaian dan kecepatan lewat ikan yang melewati pipa sensor, mengganti 
ukuran ikan yang akan dideteksi mengharuskan tes ulang untuk 
mendapatkan pengaturan alat yang paling optimal (dalam tes, Rusyidi 
menemukan bahwa kelandaian pipa 30° memberikan hasil paling akurat dalam 
mendeteksi ikan). Sementara pada alat Al-Amri, lubang keluar 
ikan yang perlu dimodifikasi untuk mengamodasi ikan yang lebih besar. 
Metode-metode tersebut sangatlah tidak fleksibel dalam industri peternakan 
ikan yang tidak hanya menternakan satu jenis ikan saja.

% Penjelasan Rapid Object Detection
Deteksi Objek Cepat (\emph{Rapid Object Detection}) adalah sebuah algoritma yang 
diciptakan untuk pendeteksian muka \cite{violaetal}. Viola 
menjelaskan kalau algoritma deteksi yang diciptakannya dapat mendeteksi 
muka dari gambar berukuran 384 x 288 pixel dari kamera berkecepatan 15 
\textit{frame} per detik. Deteksi objek dapat digunakan untuk berbagai hal seperti 
anotasi gambar, penghitungan mobil, deteksi muka, rekognisi muka, pelacakan 
gambar dan lain-lain. Setiap algoritma deteksi objek bekerja dengan cara yang 
berbeda-beda namun dengan konsep yang kurang lebih sama. Setiap kelas objek 
pasti memiliki fitur yang dapat menunjukan jati diri objek tersebut, misalnya 
objek bola sepak pastilah bulat dan umumnya memiliki dua warna yaitu hitam dan 
putih dengan pola yang spesifik. Muka manusia memiliki mata, hidung dan mulut 
yang dapat dibedakan dengan mahluk lainnya misalnya dengan kucing. Metode-metode 
deteksi objek umumnya menggunakan pendekatan neural network dan 
\textit{non-neural network} untuk mendefinisikan fitur dari kelas objek yang 
berusaha dideteksi.

% Metode Adaboost
Adaboost (\cite{freundetal}) adalah sebuah pendekatan \textit{non-neural network} 
yang sering digunakan untuk mendefinisikan fitur dari objek yang ingin 
dideteksi (\cite{weber}). Adaboost telah digunakan untuk deteksi berbagai objek 
seperti deteksi plat nomor kendaraan bermotor (\cite{hoetal}), deteksi muka 
(\cite{violaetal}), deteksi pesawat terbang (\cite{freundetal}) dan lain-lain. 
Adaboost mencari fitur sebuah kelas objek dengan menggunakan sekumpulan 
\emph{weak learner} untuk membuat sebuah \emph{strong learner}. 
Kumpulan weak learner tersebut nantinya akan dinilai sesuai dengan akurasi 
mereka, di mana weak learner yang secara konsisten benar menebak fitur sebuah 
objek akan memiliki nilai lebih dalam keputusan klasifikasi akhir.

% Usulan
Berdasarkan latar yang telah dijelaskan, penulis mengusulkan untuk mendeteksi 
jenis ikan dengan menggunakan metode \emph{Viola-Jones Object Detection Framework}. 
\emph{Viola-Jones Object Detection Framework} berguna untuk 
mengkonstruksi sebuah \emph{classifier} objek ikan yang nantinya dapat mendeteksi 
ikan dalam input gambar maupun video. Hasil yang diharapkan adalah sistem 
mampu mengklasifikasi ikan dari gambar secara akurat.

\section{Rumusan Masalah}
Dari uraian permasalahan di atas, perumusan masalah dalam penelitian ini adalah 
'\textbf{Bagaimana caranya mengklasifikasi ikan menggunakan metode 
\textit{Viola-Jones Object Detection Framework}?}'.

\section{Batasan Masalah}
Batasan masalah pada penelitian ini adalah:
\begin{enumerate}
	\item Klasifikasi ikan menggunakan \textit{Viola-Jones Object Detection Framework}.
	\item Klasifikasi harus bisa melakukan klasifikasi tiga kelas genus ikan, Abudefduf, Amphiprion, Chaetodon, 
	dan satu kelas negatif.
	\item Klasifikasi dilakukan dengan gambar tampak samping ikan saja.
\end{enumerate}

\section{Tujuan Penelitian}
	Tujuan dari penelitian ini adalah menmbuat program yang mampu
	mengklasifikasi ikan dengan menggunakan \textit{Viola-Jones Object Detection Framework}.

\section{Manfaat Penelitian}
\begin{enumerate}
	\item Bagi penulis
		
	Memperoleh gelar sarjana dalam bidang Ilmu Komputer, serta menambahkan 
	pengalaman dalam pembuatan sebuah program komputer dengan aplikasi dunia 
	nyata. Dan menambahkan pengetahuan penulis tentang deteksi objek, metode 
	\textit{Adaboost} dan \textit{Harr-Like Features}.
		
	\item Bagi Program Studi Ilmu Komputer
	
	\begin{itemize}
		\item Mahasiswa
		
		Diharapkan penelitian ini dapat digunakan sebagai penunjang referensi, 
		khususnya pustaka tentang klasifikasi object dengan 
		\textit{Viola-Jones Object Detection Framework}.

		\item Bagi Peneliti Selanjutnya
		
		Diharapkan penelitian ini dapat digunakan sebagai dasar atau kajian 
		awal bagi peneliti lain yang ingin meneliti permasalahan yang sama.

	\end{itemize}
			
\end{enumerate}

% Baris ini digunakan untuk membantu dalam melakukan sitasi
% Karena diapit dengan comment, maka baris ini akan diabaikan
% oleh compiler LaTeX.
\begin{comment}
\bibliography{daftar-pustaka}
\end{comment}
