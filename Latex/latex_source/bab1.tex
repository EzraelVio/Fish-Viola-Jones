%!TEX root = ./template-skripsi.tex
%-------------------------------------------------------------------------------
% 								BAB I
% 							LATAR BELAKANG
%-------------------------------------------------------------------------------

\chapter{PENDAHULUAN}

\section{Latar Belakang Masalah}

% Intro
Ikan merupakan salah satu kekayaan hayati yang dimiliki oleh bangsa Indonesia 
dalam jumlah yang sangat besar. Ikan telah menjadi sumber makanan bagi bangsa 
Indonesia dari zaman dahulu, dibuktikan dengan banyaknya resep masakan ikan yang 
ada di kuliner tradisional rakyat Indonesia. Selain itu banyak orang yang 
memelihara ikan tidak untuk dimakan melaikan untuk hiasan, seperti di dalam 
kolam maupun akuarium.

% Sumber ikan
Ikan bisa didapatkan melalui dua sumber, dengan menangkap ikan dari habitat 
aslinya dan dengan melalukan proses pembudidayaan. Pembudidayaan ikan 
memerlukan banyak infrastruktur pendukung yang tidaklah murah seperti 
lahan, konstruksi tambak atau kolam yang memadai, dan pakan dalam jumlah 
yang cukup besar. Walaupun cukup mahal, budidaya ikan dapat 
menghasilkan keuntungan yang besar. Ikan budidaya contohnya bisa dijual 
dalam keadan hidup atau segar ke wilayah-wilayah yang jauh dari habitat asli ikan, 
hal yang sulit dilakukan dengan ikan hasil tangkapan liar. Budidaya ikan 
juga dapat mengurangi permintaan pasar untuk ikan secara besar, yang secara tidak 
langsung berkontribusi terhadap berkurangnya \textit{overfishing} atau penangkapan 
ikan berlebih yang berdampak pada berkurangnya populasi ikan dalam jumlah 
yang besar di habitat aslinya. Permintaan untuk ikan sangat besar di 
Indonesia, hal ini menyebabkan besarnya nilai pasar dari industri ini. 
Kementrian Kelautan dan Perikanan Indonesia mencatat bahwa 
produksi ikan budidaya di Indonesia mencapai 14 juta ton pada tahun 
2021 dengan nilai sebesar Rp. 196.000.000.000.000,- trilliun rupiah (KKP, 2021).

% Dasar problem permasalahan lapangan
Budidaya ikan di Indonesia memiliki problem yang lumayan besar yaitu 
diperlukannya usaha yang besar untuk menghitung dan mengawasi jumlah ikan 
yang dibudidayakan. Dalam penghitungan bibit pada proses jual beli contohnya, 
dalam menghitungan bibit lele para pedagang masih menghitung ikan dengan 
cara manual (\cite{alamri}). Bibit ikan dipindahkan satu persatu atau 
ditimbang sesuai berat untuk mendapatkan jumlah ikan. Cara-cara ini anatara sangat tidak 
efisien atau kurang akurat. Dalam metode penghitungan contohnya, ikan 
dihitung satu per satu dengan tangan atau dengan bantuan sendok atau centong, 
yang memungkinkan penghitung untuk mengambil ikan dengan jumlah tertentu. 
Metode ini bisa memakai waktu yang cukup lama bilamana ikan yang dihitung 
ada dalam jumlah besar, maka dari itu metode ini biasanya hanya digunakan 
dalam menghitung ikan dalam jumlah yang sedikit. Tingkat akurasi yang 
tinggi menjadi keuntungan utama dari metode penghitungan.

Sementara itu metode penimbangan hanya menghasilkan jumlah perkiraan yang tidak 
selalu akurat. Dalam metode ini bibit ikan dimasukan ke dalam suatu wadah dan lalu 
beratnya dihitung menggunakan timbangan. Berat hasil penimbangan lalu bisa 
dijadikan acuan kira-kira jumlah ikan yang terdapat di dalam wadah. 
Metode ini cepat dan cukup efisien dalam menghitung ikan dalam jumlah yang 
sangat besar. Namun jumlah bibit ikan tidaklah akurat dan hanya berupa perkiraan belaka.

% Elaborasi lebih lanjut problem lapangan
Problem perhitungan ikan ini akan sangat terasa pada industri budidaya ikan yang 
sangat mementingkan kepadatan populasi dalam tempat budidaya ikan. Populasi ikan 
yang berlebihan dapat memperlambat pertumbuhan ikan (Diansari et al, 2013), tapi 
di sisi lain populasi ikan yang terlalu kecil akan mengurangi efisiensi lahan 
yang dimiliki peternak ikan. Dalam mengatasi problem Al-Amri (2020) menciptakan 
sebuah sistem penghitungan menggunakan sensor \emph{proximity}. Hasil uji coba mendapat 
hasil yang baik dengan persentase error sebesar 4,07\% dengan penghitungan 
memakan waktu 228 detik per 1000 ikan. Jauh lebih cepat dibanding kecepatan 
hitung manual yang memakan waktu 20 menit per 1000 ikan. Cara lain dipakai 
oleh Rusydi (2019) untuk mendeteksi ikan. Rusydi menciptakan sebuah alat 
penghitungan dengan katup otomatis yang akan terbuka bilamana jumlah ikan 
yang diinginkan telah tercapai. Alat tersebut mendeteksi ikan menggunakan 
konsep \emph{through beam} di mana ikan akan terdeteksi ketika melewati pipa oleh 
inframerah dan photodioda. Alat tersebut dapat mendeteksi ikan dengan 
kecepatan 58 ms per ikan dengan tingkat akurasi 100\%.

% Kelemahan metode penghitungan manual Al-Amri dan Rusydi
Penggunaan alat deteksi fisik seperti yang digunakan Al-Amri (2020) maupun 
Rusydi (2019) memiliki beberapa kekurangan, seperti ukuran ikan bergantung 
kepada ukuran alat yang dipergunakan. Alat Rusyidi (2019) sangat bergantung 
dengan kelandaian dan kecepatan lewat ikan yang melewati pipa sensor, mengganti 
ukuran ikan yang akan dideteksi mengharuskan tes ulang untuk 
mendapatkan pengaturan alat yang paling optimal (dalam tes, Rusyidi (2019) 
menemukan bahwa kelandaian pipa 30° memberikan hasil paling akurat dalam 
mendeteksi ikan). Sementara pada alat Al-Amri (2020), lubang keluar 
ikan yang perlu dimodifikasi untuk mengamodasi ikan yang lebih besar. 
Metode-metode tersebut sangatlah tidak fleksibel dalam industri peternakan 
ikan yang tidak hanya menternakan satu jenis ikan saja.

% Penjelasan Rapid Object Detection
Deteksi Objek Cepat (\emph{Rapid Object Detection}) adalah sebuah algoritma yang 
diciptakan untuk pendeteksian muka (Viola et al, 2001). Viola (2001) 
menjelaskan kalau algoritma deteksi yang diciptakannya dapat mendeteksi 
muka dari gambar berukuran 384 x 288 pixel dari kamera berkecepatan 15 
\textit{frame} per detik. Deteksi objek dapat digunakan untuk berbagai hal seperti 
anotasi gambar, penghitungan mobil, deteksi muka, rekognisi muka, pelacakan 
gambar dan lain-lain. Setiap algoritma deteksi objek bekerja dengan cara yang 
berbeda-beda namun dengan konsep yang kurang lebih sama. Setiap kelas objek 
pasti memiliki fitur yang dapat menunjukan jati diri objek tersebut, misalnya 
objek bola sepak pastilah bulat dan umumnya memiliki dua warna yaitu hitam dan 
putih dengan pola yang spesifik. Muka manusia memiliki mata, hidung dan mulut 
yang dapat dibedakan dengan mahluk lainnya misalnya dengan kucing. Metode-metode 
deteksi objek umumnya menggunakan pendekatan neural network dan 
\textit{non-neural network} untuk mendefinisikan fitur dari kelas objek yang 
berusaha dideteksi.

% Metode Adaboost
Adaboost (Freund et al, 1995) adalah sebuah pendekatan \textit{non-neural network} 
yang sering digunakan untuk mendefinisikan fitur dari objek yang ingin 
dideteksi (Weber, 2005). Adaboost telah digunakan untuk deteksi berbagai objek 
seperti deteksi plat nomor kendaraan bermotor (Ho et al, 2009), deteksi muka 
(Viola et al, 2001), deteksi pesawat terbang (Weber, 2005) dan lain-lain. 
Adaboost mencari fitur sebuah kelas objek dengan menggunakan sekumpulan 
\emph{weak learner} untuk membuat sebuah \emph{strong learner}. 
Kumpulan weak learner tersebut nantinya akan dinilai sesuai dengan akurasi 
mereka, di mana weak learner yang secara konsisten benar menebak fitur sebuah 
objek akan memiliki nilai lebih dalam keputusan klasifikasi akhir. Weber (1995) 
menjelaskan bahwa metode Adaboost dapat digambarkan selayaknya seorang pejudi 
pacuan kuda yang kalah terus menerus. Pejudi tersebut lalu memutuskan untuk 
menjadikan teman-teman penjudinya menjadi acuan untuk taruhan berikutnya. 
Adaboost bekerja layaknya penjudi tersebut, membuat asumsi dari berbagai 
pendapat lemah. Tentu saja bilamana sang penjudi melihat bahwa seorang 
temannya bertaruh dengan baik berulang-ulang kali, maka dia akan 
memandang tinggi pendapatnya di atas teman-teman lainnya yang tidak menang 
sebanyak orang tersebut. Adaboost juga bekerja mirip dengan situasi tersebut.

% Usulan
Berdasarkan latar yang telah dijelaskan, penulis mengusulkan untuk mendeteksi 
jenis ikan dengan menggunakan metode \emph{Viola-Jones Object Detection Framework}. 
Adaboost dalam \emph{Viola-Jones Object Detection Framework} berguna untuk 
mengkonstruksi sebuah \emph{classifier} objek ikan yang nantinya dapat mendeteksi 
ikan dalam input gambar maupun video. Hasil yang diharapkan adalah sistem 
mampu mengklasifikasi ikan dari gambar maupun video secara akurat.

\section{Rumusan Masalah}
Dari uraian permasalahan di atas, perumusan masalah dalam penelitian ini adalah 
'\textbf{Bagaimana cara mengklasifikasi ikan menggunakan metode 
\textit{Viola-Jones Object Detection Framework}?}'.

\section{Batasan Masalah}
Batasan masalah pada penelitian ini adalah:
\begin{enumerate}
	\item Klasifikasi ikan menggunakan \textit{Viola-Jones Object Detection Framework} yang didapat dari penelitian Viola-Jones (Viola et al, 2004)
	\item Program didesain untuk mengklasifikasi ikan saja
	\item Klasifikasi dilakukan dengan gambar tampak samping ikan saja
	\item Klasifikasi harus bisa melakukan deteksi tiga kelas genus ikan, Abudefduf, Amphiprion, Chaetodon, 
	dan satu kelas negatif.
\end{enumerate}

\section{Tujuan Penelitian}
	Tujuan dari penelitian ini adalah menciptakan metode yang mampu mempermudah 
	klasifikasi ikan dengan menggunakan \textit{Viola-Jones Object Detection Framework}.

\section{Manfaat Penelitian}
\begin{enumerate}
	\item Bagi penulis
		
	Memperoleh gelar sarjana dalam bidang Ilmu Komputer, serta menambahkan 
	pengalaman dalam pembuatan sebuah program komputer dengan aplikasi dunia 
	nyata. Dan menambahkan pengetahuan penulis tentang deteksi objek, metode 
	\textit{Adaboost} dan \textit{Harr-Like Features}.
		
	\item Bagi Program Studi Ilmu Komputer
	
	\begin{itemize}
		\item Mahasiswa
		
		Diharapkan penelitian ini dapat digunakan sebagai penunjang referensi, 
		khususnya pustaka tentang klasifikasi object dengan 
		\textit{Viola-Jones Object Detection Framework}.

		\item Bagi Peneliti Selanjutnya
		
		Diharapkan penelitian ini dapat digunakan sebagai dasar atau kajian 
		awal bagi peneliti lain yang ingin meneliti permasalahan yang sama.

	\end{itemize}
			
\end{enumerate}

% Baris ini digunakan untuk membantu dalam melakukan sitasi
% Karena diapit dengan comment, maka baris ini akan diabaikan
% oleh compiler LaTeX.
\begin{comment}
\bibliography{daftar-pustaka}
\end{comment}
