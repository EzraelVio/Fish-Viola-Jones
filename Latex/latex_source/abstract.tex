\chapter*{\centering{\emph{\large{ABSTRACT}}}}
\singlespacing{}

\textbf{MUHAMMAD HAFIZ HISBULLAH}, Detection of Chronic Wound Perimeter Using the 
GrabCut Algorithm. Undergraduate Thesis, Computer Science Program, Faculty of 
Mathematics and Natural Sciences, State University of Jakarta. January 2024.
\\

Chronic wounds pose a complex health issue, especially for patients with conditions 
such as Diabetes Mellitus (DM). The wound healing process involves meticulous 
assessment and effective management, yet manual methods in wound measurement often 
prove inaccurate and time-consuming. In this research, we propose the use of the 
GrabCut method in image processing to detect the perimeter of chronic wounds. 
This method has the potential to provide objective and reliable wound assessment.
By leveraging image processing technology and machine learning, we aim to overcome 
the limitations of manual methods by employing the GrabCut algorithm. The first 
step involves capturing wound images using mobile devices, followed by color 
calibration to ensure consistency. The GrabCut method is then applied to classify 
the wound area, focusing on the separation of the main object (the wound) from 
the background. This research utilizes a dataset of chronic wound images from 
previous studies and seeks to compare the results of the GrabCut method with 
reference images as ground truth. It is hoped that this research will contribute 
to the effectiveness of chronic wound assessment, expedite the management process, 
and help alleviate the financial burden on patients.
\\
\\
\textbf{Keywords}: chronic wounds, GrabCut, image processing, wound assessment, wound management.