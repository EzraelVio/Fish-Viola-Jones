\chapter*{\centering{\emph{\large{ABSTRACT}}}}
\singlespacing{}

\textbf{Nehemiah Austen Pison}, Prototype of Fish Species Detection System 
Using Viola-Jones Feature Extraction and Decision Tree Based Boosting, Computer Science Program, Faculty of 
Mathematics and Natural Sciences, State University of Jakarta. January 2024.
\\
\\

The fisheries sector in Indonesia represents a vast market with a 
value reaching IDR 2,400 trillion in 2022. However, the full potential 
of fisheries in Indonesia has not been fully realized. In the fisheries 
industry, there is an issue with fish counting and classification, where 
fish are still manually counted one by one or by using containers. Other 
methods that employ machines also have their limitations, such as the 
specificity of the fish sizes that can be detected. In this research, 
we propose a method for fish classification using the Viola-Jones Feature 
Extraction and Decision Tree-based Boosting methods to address the challenges 
of fish counting and classification for all sizes. By employing computerized 
classification, detection is not hindered by the size or different shapes of 
fish. This method also has the potential for classifying fish species. The 
first step involves training weak classifiers to classify various features of 
different fish species, followed by Boosting and the creation of a cascade to 
expedite the classification process. This classification is expected to assist 
in the fish counting process and the classification of fish for both fish 
farmers and the general public.
\\
\\
\textbf{Keywords}: Classification, Viola-Jones, Boosting, Decision Tree, Fish.