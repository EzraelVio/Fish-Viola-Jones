\chapter*{\centering{\emph{\large{ABSTRACT}}}}
\singlespacing{}

\textbf{Nehemiah Austen Pison}, Prototype of Fish Species Detection System 
Using Viola-Jones Feature Extraction and Decision Tree Based Boosting, Computer Science Program, Faculty of 
Mathematics and Natural Sciences, State University of Jakarta. January 2024.
\\
\\

The fish farming sector in Indonesia faces a significant problem in counting fish, 
with the current method involving manual counting or other less accurate techniques. 
Machine-based methods also have their challenges, such as limitations in the size 
specifications of fish that can be counted. The counting issue is substantial in 
the Indonesian fish farming industry, which places great importance on the quantity 
of fish in cultivation. In this research, the author employs the Viola-Jones 
Feature Extraction method and Boosting based on Decision Tree for fish 
classification, which can subsequently be utilized for fish counting. 
Utilizing computer-based classification as the foundation for counting 
enables the counting of fish of varying sizes and different species. The initial step 
in this method involves training weak classifiers to classify various features of 
fish classes, followed by Boosting and the creation of a cascade to expedite the 
classification process. This classification is expected to aid in fish counting 
processes and the classification of fish for both fish farmers and the general public.
\\
\\
\textbf{Keywords}: Classification, Viola-Jones, Boosting, Decision Tree, Fish.