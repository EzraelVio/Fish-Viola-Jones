\chapter*{\centering{\large{ABSTRAK}}}
\singlespacing{}

\textbf{Nehemiah Austen Pison}, \textit{Protptype System} Klasifikasi Genus Ikan 
Menggunakan \textit{Viola-Jones Feature Extraction} dan \textit{Boosting} Berbasis 
\textit{Decision Tree}. Skripsi, Program Studi Ilmu Komputer, Fakultas Matematika dan Ilmu Pengetahuan Alam, Universitas Negeri Jakarta. Januari 2024.
\\
\\
% Sektor peternakan perikanan di Indonesia memiliki problem yang lumayan besar dalam penghitungan. 
% Yaitu ikan masih dihitung satu per satu atau dengan metode-metode lainnya yang kurang akurat. 
% Metode lainnya yang menggunakan mesin juga punya problemnya tersendiri, yaitu limitasi 
% spesifikasi ukuran ikan yang bisa dihitung. Problem penghitungan ini sangat besar dalam 
% industri peternakan ikan di Indonesia yang sangat mementingkan jumlah ikan dalam budidaya. 
Sektor budidaya perikanan di Indonesia adalah 
sebuah sektor ekonomi yang bernilai besar bagi ekonomi 
masyarakat Indonesia, namun sektor ini masih memiliki problem yang lumayan besar 
dalam klasifikasi ikan.
Dalam penelitian ini, penulis menggunakan 
metode \emph{Viola-Jones Feature Extraction} dan \textit{Boosting} Berbasis 
\emph{Decision Tree} untuk melakukan klasifikasi ikan. Langkah pertama dalam metode ini adalah dengan melatih \emph{weak classifier} 
untuk mengklasifikasi beberapa fitur beberapa kelas ikan, diikuti dengan \emph{Boosting}, 
dan pembuatan sebuah \emph{cascade} untuk mempercepat proses klasifikasi. Hasil dari proses ini adalah 
sebuah program yang dapat mengklasifikasi dan menganotasi ikan berdasarkan input gambar.
\\
\\
\textbf{Kata kunci}: klasifikasi, Viola-Jones, Boosting, Decision Tree, Ikan.