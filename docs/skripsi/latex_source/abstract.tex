\chapter*{\centering{\emph{\large{ABSTRACT}}}}
\singlespacing{}

\textbf{Nehemiah Austen Pison}, Prototype of Fish Genus Classification System 
Using Viola-Jones Feature Extraction and Decision Tree Based Boosting, Computer Science Program, Faculty of 
Mathematics and Natural Sciences, State University of Jakarta. January 2024.
\\
\\

% The fish farming sector in Indonesia faces a significant problem in counting fish, 
% with the current method involving manual counting or other less accurate techniques. 
% Machine-based methods also have their challenges, such as limitations in the size 
% specifications of fish that can be counted. The counting issue is substantial in 
% the Indonesian fish farming industry, which places great importance on the quantity 
% of fish in cultivation. 
The fisheries cultivation sector in Indonesia is an 
economically significant sector for the Indonesian 
people. However, this sector still faces considerable 
challenges in fish classification. In this study, the 
author employs the Viola-Jones Feature Extraction method 
and Boosting Based Decision Tree for fish classification. 
The first step in this method involves training weak 
classifiers to classify various features of multiple 
fish classes, followed by Boosting and the creation 
of a cascade to expedite the classification process. 
The outcome of this process is a program capable of 
classifying and annotating fish based on input images.
\\
\\
\textbf{Keywords}: Classification, Viola-Jones, Boosting, Decision Tree, Fish.